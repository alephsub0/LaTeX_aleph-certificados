\documentclass[11pt,a4paper]{ltxdoc} 
\usepackage[spanish,es-noindentfirst,es-tabla]{babel}

\usepackage[utf8]{inputenc}
\usepackage[T1]{fontenc}
\usepackage{graphicx}
\usepackage{xcolor}
\usepackage{float}
\usepackage[margin=2.5cm,left=3.5cm]{geometry}
\usepackage{mathpazo}
\usepackage{changelog}

\setlength{\parskip}{0.2\baselineskip}
\renewcommand{\baselinestretch}{1.1}

\newcommand{\file}[1]{\texttt{#1}}
\newcommand{\option}[1]{\texttt{#1}}
\newcommand{\package}[1]{\texttt{#1}}

\title{\file{aleph-certificados.sty}}
\author{Proyecto Alephsub0\\ Andr\'es Merino}
\date{2023-11-27\\ Versión 0.1}

\usepackage[colorlinks,linkcolor=teal,urlcolor=teal,
   citecolor=black,bookmarks=true]{hyperref}
\usepackage{url}

\begin{document}
 
\maketitle

\begin{abstract}
    \file{aleph-certificados.sty} es un paquete creado para dar formato y automatizar la generación de certificados. Esta clase fue generada dentro del proyecto Alephsub0 (\url{https://www.alephsub0.org/}).
\end{abstract}

\tableofcontents

\section{Introducción}

El paquete \file{aleph-certificados.sty} es parte del conjunto de clases y paquetes creados por Andrés Merino dentro de su proyecto personal Alephsub0.

\section{Uso}

Para cargar el paquete se utiliza: \cs{usepackage}|{aleph-certificados}|.

Para visualizar un ejemplo puedes acceder al repositorio de GitHub de esta clase (clic \href{https://github.com/mate-andres/LaTeX_aleph-notas}{aquí}) % o buscarlo en la galería de plantilla de Overleaf (clic \href{https://www.overleaf.com/latex/templates/plantilla-para-escribir-resumenes-de-clase/mftfvjfhdcyj}{aquí}).

\subsection{Comandos}

\DescribeMacro{\numfirmas} Este comando permite establecer el número de firmas que se van a utilizar en el certificado. 

\DescribeMacro{\firmaA}
\DescribeMacro{\firmaB}
\DescribeMacro{\firmaC}
\DescribeMacro{\firmaD}


Este comando permite establecer la firma de la persona que va a firmar el certificado. El primer argumento es opcional y permite establecer el cargo de la persona que va a firmar el certificado. El segundo argumento es obligatorio y permite establecer el nombre de la persona que va a firmar el certificado.

\DescribeMacro{\nombrecurso} Este comando define el nombre del curso

\DescribeMacro{\institucion} Este comando define el nombre de la institución que organiza el curso.

\DescribeMacro{\lugar} Este comando define el lugar donde se realiza el curso.

\DescribeMacro{\fechainicio} Este comando define la fecha de inicio del curso.

\DescribeMacro{\fechafinal} Este comando define la fecha de finalización del curso.

\DescribeMacro{\fechaentrega} Este comando define la fecha de entrega del certificado.

\DescribeMacro{\duracion} Este comando define la duración del curso.

\DescribeMacro{\logoizq} Este comando permite establecer el logo de la institución que organiza el curso. El argumento opcional permite establecer el ancho del logo.

\DescribeMacro{\logoder} Este comando permite establecer el logo de la institución que organiza el curso. El argumento opcional permite establecer el ancho del logo.

Cualquier problema adicional, por favor reportarlo a\\ 
\url{mat.andresmerino@gmail.com}.

\begin{changelog}[author=Andés Merino,
    sectioncmd=\section]
    % version 0.1
    \shortversion{author=Daniel Lara,v=0.1,
    date=2023-12-25,
    changes=Primera versión en pruebas}
\end{changelog}

\newpage
\DocInput{aleph-certificados.dtx}

\end{document}
